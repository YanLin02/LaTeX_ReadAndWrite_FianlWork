\documentclass[conference]{IEEEtran}
\usepackage[UTF8]{ctex}
\usepackage{graphicx}
\usepackage{setspace}
\usepackage{times}
\usepackage{epsfig}
\usepackage{graphicx}
\usepackage{amsmath}
\usepackage{amssymb}
\usepackage{subfigure}
\usepackage{overpic}
\setmainfont{TeX Gyre Termes}

%定义引用
\newcommand{\figref}[1]{图\ref{#1}}
\newcommand{\tabref}[1]{表\ref{#1}}
\newcommand{\equref}[1]{式\ref{#1}}
\newcommand{\secref}[1]{第\ref{#1}节}


%题目信息
\title{无人机集群路径规划}
\author{71121139 孔晔
    09021232 薛沛林
    09021230 孙彦林
    71121140 杨政贤}
\date{\today}


%文章开始
\begin{document}

%题目
\maketitle


%摘要
\begin{abstract}


\end{abstract}



%关键词
\begin{IEEEkeywords}

    东南大学


\end{IEEEkeywords}

%Section 1 引言
\section{引言}



%Section 2 无人机路径规划
\section{无人机路径规划}


%Subsection 1 路径规划与轨迹规划的区别与流行算法的介绍
\subsection{路径规划与轨迹规划的区别与流行算法的介绍}


%Subsection 2 基于抽样的基本算法与基于节点的基本算法
\subsection{基于抽样的基本算法与基于节点的基本算法}

%Subsubsection 1  Dijkstra算法
\subsubsection{Dijkstra算法}

%Subsubsection 2  PRM算法
\subsubsection{PRM算法}

%Subsubsection 3  RRT算法
\subsubsection{RRT算法}


%Subsection 3 基于生物启发的优化算法
\subsection{基于生物启发的优化算法}


%Subsection 4 集群路径规划
\subsection{集群路径规划}



%Section 3 总结
\section{总结}


%Subsection 1 文章结构及论文关系
\subsection{文章结构及论文关系}


%Subsection 2 工作意义
\subsection{工作意义}


%Subsection 3 未来研究方向[1]
\subsection{未来研究方向}


%Subsection 4 欠缺之处
\subsection{欠缺之处}


%参考文献
\bibliographystyle{IEEEtran}
\bibliography{ref.bib}

\end{document}
